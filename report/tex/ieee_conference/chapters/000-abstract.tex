\begin{abstract}
    This paper explores a Plant Seedlings Classification dataset consisting of multiple images representing \textbf{12} different plant species. Several deep learning approaches are presented, including a custom CNN trained from scratch, a pre-trained CNN (ResNet-18) and a pre-trained vision transformer (vit-base-patch16-224) tuned for this classification task. To address the challenges of data scarcity and class imbalance, extensive data augmentation techniques such as random rotations, flips and color jittering are employed. Results show that transfer learning with ResNet-18 outperforms the custom model, achieving a mean F1-score (micro-averaged) of \textbf{0.96095} on the test set. The custom CNN, while slightly less accurate, still achieves a competitive F1-score of \textbf{0.92695}, demonstrating that even smaller locally trained architectures can be viable if carefully designed and thoroughly regularized. While the vision transformer model achieves a high F1-score of \textbf{0.96725}, an ensemble combining the predictions of all three models achieves the highest F1-score of \textbf{0.97103}. Finally, potential solutions are outlined, including deeper architectures, synthetic augmentation and interpretability measures, to further improve seedling classification performance.
\end{abstract}

\begin{IEEEkeywords}
    Machine learning, Image classification, Convolutional neural networks, Vision transformers
\end{IEEEkeywords}