\begin{abstract}
    This paper evaluates and compares three deep learning~(DL) approaches for plant seedlings classification using a dataset consisting of \textit{4750} images of \textit{12} different plant species. Several DL approaches were considered, including a custom convolutional neural network~(CNN) trained from scratch, a pre-trained CNN ``ResNet-18'' and a pre-trained vision transformer~(ViT) ``vit-base-patch16-224'' fine-tuned for the task at hand. To address the challenges of data scarcity and class imbalance, extensive data augmentation techniques such as random rotations, flips and color jittering were employed. Results showed that transfer learning with ResNet-18 outperforms the custom model, achieving a mean F1-score~(micro-averaged) of \textbf{~0.961} on the test set. The custom CNN, still achieved a competitive F1-score of \textbf{~0.927}, demonstrating that even smaller locally trained architectures can be viable if carefully designed and thoroughly regularized. While the ViT model achieved the highest F1-score of \textbf{~0.967}, an ensemble combining the predictions of all three models outperformed the single models with a score of \textbf{~0.971}. Finally, potential improvements are outlined, including deeper architectures, synthetic image generation and interpretability measures, to further improve seedling classification performance.
\end{abstract}

\begin{IEEEkeywords}
    Machine learning, Image classification, Convolutional neural networks, Vision transformers, Transfer learning
\end{IEEEkeywords}